\documentclass[12pt]{IEEEconf}
\usepackage{graphicx}
\usepackage{hyperref}
\nonumber
\begin{document}
\title{The Cup Game}
\author{Henry Pick\\ Miles Christensen}
\maketitle
\section{Introduction}
\subsection{History and Motivation}
\textit{The Cup Game} originated as a nonsensical competitive cup-stacking game played by members of the CMS Cross Country Team in the Hoch Shanahan Dining Commons. While the exact time and circumstances of its origin are not documented, most people estimate the game originated some time in the early 2010's during the heyday of team dining. General popularity of \textit{The Cup Game} has fallen in recent years but it remains an integral part of the team culture and the experience of eating in the Hoch Shanahan Dining Commons. As of 2022, \textit{The Cup Game} was appropriated by other students at Harvey Mudd College who play their own variations from the canonical \textit{Cup Game}. While the importance and overall goal of participating in \textit{The Cup Game} are largely up for personal interpretation, the fundamental objective is to stack your cup on that of another participant in an action known as cupping. Due to the ambiguous nature of what it means to cup an opponent, disagreements are all too frequent without the existence of a formal set of \textit{The Cup Game} rules. Members on \textit{The Cup Game} Board have thus convened to draft and ratify a specification to shift the focus of \textit{The Cup Game} away from petty disputes and towards competitive gameplay.

\subsection{How to Use This Document}
This document is multi-purpose. It is guide, a constitution and a documentation of history. Those who are new to \textit{The Cup Game} are encouraged to read it cover-to-cover, referencing the section on terminology (\ref{section:terminology}) when necessary. When one wishes to settle ambiguities in gameplay or a dispute over the rules, they may cite the gameplay section (\ref{section:gameplay}). In the inevitable circumstance that this book does not clearly settle disputed gameplay, participants are encouraged to extend the rulebook with oversight from the cup game committee. In order to maintain an account of the rich and epic history of this game, the authors of this document are also allowed to provide accounts of landmark events in \textit{The Cup Game}'s evolution through time.

\section{Gameplay Overview}
\label{section:gameplay}
\subsection{Cupping} Cupping is the elementary action in \textit{The Cup Game}. It is essential that every player understands the circumstances under which valid cupping takes place, which are fortunately quite simple. A player in possession of a stack can only cup a viable opponent in possession of a viable stack. If these conditions are not met and a cup is attempted, the violating player must take back their stack. Additionally, no grace period is issued if the offensive cupping player fumbles while attempting to cup someone else.

The top cup on a player's stack must also be empty in order for the cupping action to take place. If this condition is not met and a cup is attempted, the violating player must empty the contents of their cup (either through consumption or by throwing it away) and place their opponents stack on top of their own cup.

A player becomes active either when they make their first cupping attempt or when an opponent successfully cups them. Once a player is active, they may not refill the top cup on their stack above the line, nor may they deposit their stack in the dish conveyer and return to their table. They may only exit \textit{The Cup Game} by successfully cupping an opponent or by forfeiting through cup duty.
\subsection{Defensive Actions}
\subsubsection{Defensive Undercupping}
\label{section:defensive_undercupping} 
Defensive undercupping as a method of cupping an opponent who is attempting an offensive cup. Its primary advantage is that in defensive circumstances, it allows a player to cup another player without becoming active themselves. It is also relatively infrequently practiced, meaning offensive cuppers are less likely to anticipate a defensive undercup. The gameplay of an undercup is as follows: a player in the act of being cupped can yell ``undercup'', after which they thrust their cup upward into the base of the stack of the offensive cupper. There must be a clear cupping attempt on behalf of the offensive cupper in order for this sequence to be valid. The undercupper must also clearly move their cup upward and cannot simply initiate an undercup verbally. If there was no clear cupping attempt on behalf of the offensive party, then the undercup manoeuver is considered offensive and will make the undercupper active.   
\subsubsection{Traditional Abstinence} Many Hoch goers find \textit{The Cup Game} a great inconvenience to their gustatory pleasures in the Commons. They look for a means of enjoying the company of their Cross Country friends without game-induced anxiety or the fear of embarrassing themselves in front of the large NARP audience that sits adjacent to the team table.

Abstinence can be practiced which protects a viable but inactive player from being cupped under all circumstances. Traditional abstinence is performed by placing the empty cup on any one of its nine sides before becoming active as a player.

It is encouraged that active game participants not berate those who choose abstinence. Those practicing abstinence should be aware, however, that these actions do have inevitable social implications (See \ref{section:social}).

\subsubsection{Cubing Abstinence}
Cubing is another excellent solution to the defensive gameplay problem that has been adopted into the formal set of \textit{The Game's} rules in order to promote a more inclusive eating culture alongside game participants.

To cube, simply fill a cup with enough ice such that the level is above the line. A finished drink will never be viable, as there is always ice obstructing a legal cup. This makes cubing a safer means of abstinence; there is no period during which the cup is viable.

\subsection{Personal Biases} 
Players are required to dissociate personal biases from their game play whenever in competition with a \textit{viable} opponent. If it can be proven that someone allowed personal bias to influence their gameplay or tactics, they may receive penalties in accordance with the severity of the bias. Additionally, it is a proven fact that personal biases tend to decrease a participant's performance when playing with a large group of competitors. Having a attentive commitment to playing \textit{The Cup Game} is generally the only thing that prevents a player from getting cupped. Those who are negligent and try to substitute quality gameplay with politics tend to suffer in the long-run.

\subsection{Miscellaneous Gameplay}
\subsubsection{Thunder God} a player in possession of a stack of two or more cups can perform the Thunder God ritual to reduce their stack size by one cup. This action may only be used once per game. In order to Thunder God, a player with more than one cup in their stack must first stand on their table and yell ``I am the Thunder God'' such that the phrase is audible to every other person eating in the Hoch. Following this, they must blow down the gap in between the top two cups on their stack hard enough for the top cup to float completely out of the stack (ie. there must be no overlap between the top and second cups in the stack). In this moment, the player grabs the top cup out of the air and discards the cup.

\subsubsection{Offensive undercupping} As discussed in \ref{section:defensive_undercupping}, offensive undercupping is the act of undercupping without a clear offensive attack on behalf of the cupped party. Historically, the difference between defensive and offensive undercupping is one of the most highly disputed distinctions in the game. It is therefore considered an advanced and ineffective tactic for beginner players.

\subsection{Terminology}\label{section:terminology}
\begin{enumerate}
    \item \textit{abstinence}---abstaining from \textit{The Cup Game} by putting your cup down before becoming active
    \item \textit{active} (player)---a player who has entered \textit{The Cup Game}
    \item \textit{board}---the advisory board of \textit{The Cup Game}, tasked with interpreting legislation, revising the rule book, and settling disputes
    \item \textit{cup} [n.]---a nonagonal cup from the Hoch (\ref{fig:cup})
    \item \textit{cup} [v.]---the act of placing your stack on another opponent's stack, thus relieving you of cup duty
    \item \textit{cup duty}---when a stack's owner puts their stack in the dish conveyer, removing the cups out of game play and terminating the owner's session on a loss
    \item \textit{cupping attempt}---any attempt to offensively cup an opponent. Defensive undercupping is not a form of cupping attempt
    \item \textit{cubing}---A form of cup game abstinence that involves filling your cup with ice above the line
    \item \textit{empty} (cup)---a cup used during a meal that now only contains liquid residue. Any solids obstructing the line invalidate the empty status
    \item \textit{loss}---when a stack's owner chooses to perform cup duty and leaves the Hoch
    \item \textit{the line}---the protrusions around half-way up the inside of a cup (\ref{fig:line})
    \item \textit{the game}---slang for \textit{The Cup Game}, not to be confused with \textit{The Game} (Fuck, I just lost the game)
    \item \textit{the Hoch}---The Hoch Shanahan Dining Commons
    \item \textit{the rulebook}---this book
    \item \textit{stack}---any nonzero number of cups stacked on their lines. Note that a stack of size one and a cup are effectively the same thing. In these definitions and in general speak, we use `cup' if and only if the object is a singular cup.
    \item \textit{undercupping}---defensively or offensively cupping a viable opponent's stack by thrusting your cup beneath theirs
    \item \textit{viable} (stack)---any time that the liquid level in the top cup of a stack is below the line, which makes it eligible for cupping
    \item \textit{viable} (opponent) --- someone who who can be cupped when their stack is viable, when they
          \begin{enumerate}
              \item are in the Hoch
              \item are consuming a beverage out of one of the Hoch's nonagonal cups
              \item have a basic understanding of \textit{The Cup Game} rules, to the point which they have the capacity to cup without violating rules outlined in Cupping and Defensive Actions
          \end{enumerate}
    \item \textit{win}---when a stack's owner cups an opponent
          \begin{figure}
              \begin{center}
                  \includegraphics[width=0.7\linewidth]{fig/line.png}
                  \label{fig:line}
                  \caption{The line in a cup}
              \end{center}
          \end{figure}
          \begin{figure}
            \begin{center}
                \includegraphics[width=0.7\linewidth]{fig/cup.png}
                \caption{The nonagonal design of a cup at the Hoch}
                \label{fig:cup}
            \end{center}
          \end{figure}
\end{enumerate}
\section{Settling Disputes}
The proliferation and complexity of disputes in the Game necessitate a dedicated protocol and committee to deal with them in an efficient and unbiased manner. The highest order of legislation is of course this rulebook, but in the case of unforeseen circumstances or semantic inconsistencies, these rules establish a means to amending the original rule set to reflect the needs of an ever-changing constituent.
\subsection{Types of Disputes}
We categorize disputes by the intention of the parties involved, the nature of the disputed action, and the level of response that is necessary on behalf of the advisory board
\begin{enumerate}
    \item Petty disputes---these types of disputes are clearly resolved by rules in this book and should be interpretable by any lay reader. However, many of these disputes arise to reverse the outcome of gameplay by a frustrated player and therefore require a single member of the advisory board to dictate the ruling of the dispute.
    \item Misunderstandings---these types of disputes arise when a player does not understand the rules, resulting in a genuine conflict between players over some action. These types of disputes can also be resolved by reading the rules directly. Since they result from genuine misunderstandings, it is recommended that the players consult the rule book first before appealing to a board member for a settlement.
    \item Aggravated disputes---arising in any situation in which members of the game are put in imminent danger at the hands of an enraged player of the game. The rule book's legislation is almost always irrelevant in these disputes. Instead, members of the advisory board must intervene to physically diffuse the situation.
    \item Novel disputes---arising from any situation in which the rulebook does not clearly determine the legality of certain gameplay. In these cases, a cup game board meeting is necessary and legislation may be revised.
\end{enumerate}

\subsection{Penal Measures}
The game's rules are self-enforcing for the most part, but in certain cases, active punishment is necessary to maintain justice among the community of participants. The most common violation that has little consequence through actual gameplay is the influence of personal bias. When it is proven that a player has allowed personal bias to dictate their actions in the game, they will be subject to the three-strikes policy, which is as follows:
\begin{enumerate}
    \item On the first instance, an offending player will receive a warning from the Cup Master and they may be requested to recall their last action
    \item On the second offense, one must write a one-page written reflection on the motivation and consequences of their violation and how it degrades the culture of trust within the participant community. They submit this response to Cup Game Board for review and a potential follow-up
    \item On the third offense, the player will be banned from any offensive action or undercupping for a month following the offense. Note that they can still be cupped
\end{enumerate}

A comprehensive list of offenses that are handled through the three-strikes policy are given below
\begin{enumerate}
    \item Personal bias influence in gameplay
    \item Intention of exclusion--any action made to exclude or otherwise prohibit members of the Cross Country team from participating in the game
    \item Playing the game with NARPS
    \item Claiming exemption from rules or the Cup Game in general
\end{enumerate}

There are more severe measures designed to handle extreme cases of misconduct. They are generally less forgiving than the three strikes rule and apply indefinitely
\begin{enumerate}
    \item Those who are convicted of motivated physical attack are banned from sitting at the Cross Country table. You should probably also handle this with the school disciplinary board if it happens lol.
    \item Any form of verbal assault results in a ban from sitting at the Cross Country table 
    \item Those who deny the authority of, or otherwise disrespect members of The Cup Game Board and the Cup Master are banned from any form of abstinence
\end{enumerate}
\subsection{Consulting the Rules}
If you find yourself needing to consult the rulebook, either to inform yourself on gameplay or to settle a dispute, you will find that the rulebook is both a fun cover-to-cover read and an easy-to-navigate document complete with a linked PDF outline and hyperlink references. All game-specific vocabulary is documented in \ref{section:terminology}, but if clarifications in the text are needed, do not hesitate to bring a request for addendum to an advisory board member.

When consulting the rules in the event of a dispute, it is recommended that you cite directly from the text. There should be no extended interpretations of this book, since it was written with the intent of being a literal prescription of rules.

In its small format, these rules should be portable and can be carried by a holder in the Hoch. If you are so inclined, feel free to print out the original copy, found \href{https://www.cs.hmc.edu/~hpick/Cupgame%20Rules.pdf}{here}. Counterfeits can always be checked against this online copy.
\subsection{Amendments}
The amendment process is the sole procedure through which the rule book is changed. The procedure is made as simple as possible and is hopefully progressive enough to permit reasonable amendments with near certainty. We outline it as follows:
\begin{enumerate}
    \item Proposal is made to the Cup Game Board
    \item Cup Game Board reviews proposals when convenient. A simple 50\% vote is required to approve the proposal
    \item Cup Game Board drafts an amendment from a proposal and then votes to ratify with 75\% majority
\end{enumerate}
Once an amendment is made, it must be committed to the rule book, git repo hosted \href{https://github.com/henry-2025/Cupgame-Rules}{here}. Pushing to main is fine.
\section{Social Implications}
\label{section:social}
When playing the Cup Game, one must understand the inevitable social consequences that arise. These are immediately evident in your interactions with other players, which may become tense reflecting disputes during gameplay. It is in everyone's best interest to limit the extent of these feelings because they are often what lead to influence of personal bias. One must also understand that there is a stigma towards players who choose abstinence. This is something that the Cup Game Board wilfully does not protect against, as abstinence has led to a large fall-off in game participation in recent times. As with all things, forms of prejudice towards ``abstintees'' are not tolerated if they are overtly disrespectful.

As it pertains to your interactions with the greater 5C community, you will be seen as loud and disrespectful from the perspective of most other people sitting in the Hoch. But honestly, who gives a shit. In some cases, non-team members may try to copy the Cup Game, stacking cups at their tables without form or fashion. If you do see this happening, there may be a desire to inform them that they are mocking a sacred tradition and that they should have some respect for you because you run Cross Country. Again, they probably don't give a shit and you are better off not wasting your energy on that interaction. The social relationship between the canonical Cup Game and knock-off versions is meant to be one of coexistence. This game is not intended for a popular audience and one should not try to make it such.

\section{Admin and Legislation}
\subsection{The Board of Cuppers and The Cupmaster}
A lineage of cup game adjudicators exists that is self-selecting and a large part of the game's tradition on the team. The Cup Master was initially sole member of this committee, assuming all responsibilities of legislation, punitive enforcement, and communication of the rules. The typical term of the Cupmaster is 4 years, or the duration of their attendance at the 5Cs. The tradition is for the graduating Cupmaster to appoint one of the frosh in the Spring semester. In 2023, a Cup Game Board was designed to help distribute responsibilities and create a more diverse legislative body that was not viewed as monarchist. The Cup Game Board comprises of an additional 3 elected officials who are nominated by the Cross Country team and approved by the standing Cupmaster. The duration of board members is more fluid; they can resign at their own will, after which new elections are held. In relation to the board the Cup Master still plays the essential role of a spiritual/cultural guide for the Cup Game on CMS Cross Country while the rest of the Cup Game Board just legislative and judiciary. A list of past Cupmasters (and advisory boards if  applicable) is given below
\begin{center}
    \includegraphics*[width=0.7\linewidth]{fig/wilson.jpg}

    Wilson Ives, Cupmaster 2016-2020

    \includegraphics*[width=0.7\linewidth]{fig/henry.png}

    Henry Pick, Cupmaster (self-proclaimed) 2020-2023

    \includegraphics*[width=0.7\linewidth]{fig/jabro.png}

    Matthew Jabro, Cupmaster elect 2023-
\end{center}
\section{Landmark Events}
\textit{Important Events Inside the Hoch that may or may not be related to the Cup Game}

\begin{itemize}
    \item 15 Sept. 2021---Adam ``Champ'' Wilkinson ('22) famously says, ``Don't talk to me, I run cross country'' while standing in line for Hoch Wet Burrito. The statement is overheard in the Hoch and becomes a sensation on YikYak
    \item 19 Mar. 2023---Marin Muncan Oly. attends a Hoch wet burrito team meal
\end{itemize}
\section{Amendments}
\end{document}